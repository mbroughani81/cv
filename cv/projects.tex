%-------------------------------------------------------------------------------
%	SECTION TITLE
%-------------------------------------------------------------------------------
\cvsection{Projects and Studies}


%-------------------------------------------------------------------------------
%	CONTENT
%-------------------------------------------------------------------------------
\begin{cventries}

%---------------------------------------------------------
  % \cventry
  % {Computer Networks Project} % Organisation
  % {Torrent} % Project
  % {Tehran, Iran} % Location
  % {Feb 2023 - Mar 2023} % Date(s)
  % {
  %   \begin{cvitems} % Description(s) of project
  %     \item {Link: \href{https://github.com/mbroughani81/Torrent-NetworkCourse/}{\textit{https://github.com/mbroughani81/Torrent-NetworkCourse/}}}
  %     \item {As one the projects of Computer Networks course, we developed an app with features similar to Torrent, in which a couple of nodes connect to a master node and each one can download files, share its files with other nodes.}
  %     \item {\textbf{Technical Skills:} Socket Programming - Multithreading}
  %   \end{cvitems}
  % }

  \cventry
  {Personal Project} % Organisation
  {jlox Language Implementation} % Project
  {Tehran, Iran} % Location
  {Feb 2023 - Mar 2023} % Date(s)
  {
    \begin{cvitems} % Description(s) of project
    \item {Link: \href{https://github.com/mbroughani81/jlox}{\textit{https://github.com/mbroughani81/jlox}}}
    \item {As a suggestion of a graduate student from my university, I read the
        \href{https://craftinginterpreters.com/}{\textit{Crafting Interpreters}}
        book to start my studies in programming languages theory. Following along the book, I learned how languages are implemented and the design decisions involved in the process.}
    \item {\textbf{Technical Skills:} parsing, context-free grammars, interpreter design, programming language design}
    \end{cvitems}
  }

  \cventry
  {Personal Study} % Organisation
  {Extracurricular Self-Study} % Project
  {} % Location
  {} % Date(s)
  {
    \begin{cvitems} % Description(s) of project
\begin{itemize}
        \item  Formal Verification: \newline
          I attended the recorded \href{https://www.youtube.com/playlist?list=PLwabKnOFhE38C0o6z_bhlF_uOUlblDTjh}{\textit{"Model Checking"}} course by Prof. Katoen. In this course I learned how we can use Model Checking methods to verify some behavior in complex software systems.
          \newline
          I also read about the basics of LEAN proof assistant and how to formalize program semantics from The \href{https://github.com/blanchette/logical_verification_2023}{\textit{Hitchhiker’s Guide to Logical Verification}} book.
        \item  Concurrency: \newline
          Reading
          \href{https://pragprog.com/titles/pb7con/seven-concurrency-models-in-seven-weeks/}{\textit{Seven
              Concurrency Models in Seven Weeks}} helped me to learn about
          different ways we can enhance performance in software systems and
          pros and cons of each of these approaches.
        \item Computer Systems:
          \newline
          After reading
          \href{http://csapp.cs.cmu.edu/2e/samples.html}{\textit{
              Computer Systems: A Programmer's Perspective}}, I gained
          in-depth knowledge
          about aspects of computer systems and how they can be used to achieve high performance in software.
        \item Programming Languages:
          \newline
          Books like \textit{SICP} and \textit{Crafting Interpreters} deepened knowledge in programming languages theory.
        \end{itemize}
    \end{cvitems}
  }
  % ---------------------------------------------------------
%  \cventry
%    {Personal Study} % Organisation
%    {Reading about programming language theory and concurrent programming} % Project
%    {Tehran, Iran} % Location
%    {Oct 2022 - Current} % Date(s)
%    {
%      \begin{cvitems} % Description(s) of project
%       \item {After encountering the software architecture challenges and working with different programming paradigms in the Rahnema College bootcamp, I decided to study more about programming language theory and the solutions they provide for software development. As a suggestion from bootcamp teachers, I read the book \href{https://leanpub.com/fp-made-easier}{\textit{Functional Programming Made Easier}} and got many useful and practical ideas about software development. Recently I got interested in learning about concurrency and the pros and cons of the way that different programming languages support this facility. An interesting book that I read while doing this study is \href{https://pragprog.com/titles/pb7con/seven-concurrency-models-in-seven-weeks/}{\textit{Seven Concurrency Models in Seven Weeks}}. After reading this book, I got a grasp of solutions for developing concurrent programs and their use cases.}
%      \end{cvitems}
%    }


%---------------------------------------------------------
\end{cventries}
