%-------------------------------------------------------------------------------
%	SECTION TITLE
%-------------------------------------------------------------------------------
\cvsection{Projects and Studies}


%-------------------------------------------------------------------------------
%	CONTENT
%-------------------------------------------------------------------------------
\begin{cventries}

%---------------------------------------------------------
  \cventry
  {Computer Networks Project} % Organisation
  {Torrent} % Project
  {Tehran, Iran} % Location
  {Feb 2023 - Mar 2023} % Date(s)
  {
    \begin{cvitems} % Description(s) of project
      \item {Link: \href{https://github.com/mbroughani81/Torrent-NetworkCourse/}{\textit{https://github.com/mbroughani81/Torrent-NetworkCourse/}}}
      \item {As one the projects of Computer Networks course, we developed an app with features similar to Torrent, in which a couple of nodes connect to a master node and each one can download files, share its files with other nodes.}
      \item {\textbf{Technical Skills:} Socket Programming - Multithreading}
    \end{cvitems}
  }

  \cventry
    {Personal Project} % Organisation
    {jlox Language Implementation} % Project
    {Tehran, Iran} % Location
    {Feb 2023 - Mar 2023} % Date(s)
    {
      \begin{cvitems} % Description(s) of project
        \item {Link: \href{https://github.com/mbroughani81/jlox}{\textit{https://github.com/mbroughani81/jlox}}}
        \item {As a suggestion from a graduate student about his field - programming languages - I started to read the \href{https://craftinginterpreters.com/}{\textit{Crafting Interpreters}} book. Following along the book, I read the first half of the book, which was about implementing a tree-walk interpreter written in Java. Up to this point in the book, I got better a better insight into the way that languages are implemented and the design decisions that are made in the meanwhile. I will soon start the other half of the book, which is about implementing the same language(in C language), but this time by implementing a virtual machine and writing a compiler to generate bytecodes.}
        \item {\textbf{Technical Skills:} Parsing - Context Free Grammars - Interpreters design - Programming language design}
      \end{cvitems}
    }

\cventry
{Personal Study} % Organisation
{Reading about interesting topics in CS} % Project
{Tehran, Iran} % Location
{Oct 2022 - Current} % Date(s)
{
  \begin{cvitems} % Description(s) of project
    \item {In order to deepen my knowlege, I always look for interesting resources to read.
      Some of the examples:
      \begin{itemize}
        \item  Formal Verification: \newline
                Attended recorded \href{https://www.youtube.com/playlist?list=PLwabKnOFhE38C0o6z_bhlF_uOUlblDTjh}{\textit{"Model Checking"}} course by Prof. Katoen. In this course I learned how we can use Model Checking methods to verify some behaviour in a complex software systems.
                \newline
                I also read about the basics of LEAN proof assisstant and how to formalizing program semantics from The \href{https://github.com/blanchette/logical_verification_2023}{\textit{Hitchhiker’s Guide to Logical Verification}} book.
        \item  Concurrency: \newline
                Reading \href{https://pragprog.com/titles/pb7con/seven-concurrency-models-in-seven-weeks/}{\textit{Seven Concurrency Models in Seven Weeks}} helped me to learn about different ways we can use concurrency in software engineering.
        \item Computer Architecture:
                \newline
                After reading
        \href{http://csapp.cs.cmu.edu/2e/samples.html}{\textit{
            Computer Systems: A Programmer's Perspective}} I learned a lot about important ascpects of hardware design and how they can be used to achieve high performance in software.
        \item Programming Languages:
                \newline
                Interesing books like \textit{SICP} and \textit{Crafting Interpreters} helpled me to get deep knowledge in programming languages theory.
      \end{itemize}
    }
  \end{cvitems}
}

%---------------------------------------------------------
%  \cventry
%    {Personal Study} % Organisation
%    {Reading about programming language theory and concurrent programming} % Project
%    {Tehran, Iran} % Location
%    {Oct 2022 - Current} % Date(s)
%    {
%      \begin{cvitems} % Description(s) of project
%       \item {After encountering the software architecture challenges and working with different programming paradigms in the Rahnema College bootcamp, I decided to study more about programming language theory and the solutions they provide for software development. As a suggestion from bootcamp teachers, I read the book \href{https://leanpub.com/fp-made-easier}{\textit{Functional Programming Made Easier}} and got many useful and practical ideas about software development. Recently I got interested in learning about concurrency and the pros and cons of the way that different programming languages support this facility. An interesting book that I read while doing this study is \href{https://pragprog.com/titles/pb7con/seven-concurrency-models-in-seven-weeks/}{\textit{Seven Concurrency Models in Seven Weeks}}. After reading this book, I got a grasp of solutions for developing concurrent programs and their use cases.}
%      \end{cvitems}
%    }


%---------------------------------------------------------
\end{cventries}
